The industrial sector is responsible for a large share of worldwide energy consumption. The reduction of energy consumption by the industrial sector can lower the rate and magnitude of future climate change impacts on people and ecosystems. Non-intrusive load monitoring (NILM) is an algorithm that disaggregates the aggregate power consumption of a facility into the specific power consumption of the facility's equipment. NILM supplies information that can lead to strategies for optimal energy usage in a facility and therefore decrease the energy demand in the industrial sector. The dissertation aims to develop a NILM algorithm to be part of an intelligent platform for the management of microalgae production, supported by probabilities and optimization algorithms, within the scope of the InGestAlgae project (reference: CENTRO-01-0247-FEDER-046983) develop at the Institute of Systems and Robotics (ISR) of the University of Coimbra.

% add information about the results (percentage)
% add information about the methodology
% add information about the conclusions

% add information about SCADA

\textbf{Keywords:} event-based, low-frequency, unsupervised, equipment state, SCADA